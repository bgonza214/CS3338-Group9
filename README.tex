\documentclass[11pt]{article}

\usepackage[margin=1in]{geometry}
\usepackage{setspace}
\usepackage{hyperref}
\usepackage{enumitem}
\usepackage{graphicx}


\usepackage{helvet}
\renewcommand{\familydefault}{\sfdefault}

\setstretch{1.15}

\title{README \& User Manual\\[0.25em]
SCE VR Training Program}
\author{Brailey Gonzalez-Oxlaj, Derek Rosales ,Kevin Guerrero}
\date{\today}

\begin{document}

\maketitle

\section{Project Overview}

\subsection{Formal Objective}
The SCE VR Training Program is a virtual reality training application designed for
Southern California Edison (SCE). The objective of the project is to provide an
immersive, step-by-step training experience that simulates real-world overhead
maintenance scenarios on electric poles. Trainees can safely practice installation and
maintenance procedures for Power Delivery Products (PDP) such as smart
navigators and related equipment without risking injury or equipment damage.

\subsection{Why This Project Is Needed}
SCE frequently works with both internal field workers and third-party contractors.
Contractors often do not receive the same depth of training as full-time employees.
Traditional training methods can be expensive, constrained by physical resources,
and potentially hazardous.

The VR Training Program:

\begin{itemize}[nosep]
  \item Provides a safe, controlled training environment.
  \item Closely mimics real-world tools, equipment, and procedures.
  \item Reduces costs associated with in-person training and equipment wear.
  \item Increases trainee confidence and familiarity before performing field work.
  \item Helps standardize training across internal staff and contractors.
\end{itemize}

\subsection{Key Features}
\begin{itemize}[nosep]
  \item Immersive VR simulation using Meta Quest 2.
  \item Guided, step-by-step training flow for overhead installation tasks.
  \item In-world UI with ``Next'', ``Previous'', and audio playback controls.
  \item Inventory system that allows users to carry tools between scenes.
  \item Audio feedback and 3D sound for immersion.
  \item Haptic feedback when interacting with correct objects.
  \item LED sequence simulation for Pole Master device states.
\end{itemize}

\section{Repository and Project Management}

\subsection{Git Repository}
The project source code and assets are hosted in a Git repository:

\begin{quote}
\texttt{https://github.com/bgonza214/CS3338-Group9} % TODO
\end{quote}

Clone the repository using:

\begin{verbatim}
git clone https://github.com/bgonza214/CS3338-Group9.git
\end{verbatim}

\subsection{Jira Board}
Project tasks, sprints, and user stories are tracked on Jira:

\begin{quote}
\url{https://calstatela-team-group9.atlassian.net/jira/software/projects/SCRUM/boards/1}
\end{quote}

Refer to the Jira board for current sprint status, open issues, and backlog items.

\section{System Requirements}

\subsection{Hardware Requirements}

\begin{itemize}[nosep]
  \item \textbf{VR Headset:} Meta Quest 2.
  \item \textbf{Controllers:} Meta Quest 2 left and right controllers.
  \item \textbf{Development PC (for building and deploying):}
  \begin{itemize}[nosep]
    \item 8 GB RAM or higher.
    \item Processor: Intel i5-4590 / AMD Ryzen 5 1500X or greater.
    \item At least 1x USB port.
    \item Stable internet connection (for package downloads and deployment).
  \end{itemize}
\end{itemize}

\subsection{Software Requirements}

\begin{itemize}[nosep]
  \item Windows 10 or higher (Version 22H2 or later recommended).
  \item Unity Editor (2021.3.x LTS; originally 2021.3.5f1).
  \item Unity Hub.
  \item Meta Quest OS version 46.0 or higher on the headset.
  \item Android build support installed in Unity (including SDK, NDK, and OpenJDK).
  \item Visual Studio 2019 or later with C\# support.
\end{itemize}

\section{Downloading and Building the Project}

\subsection{Cloning the Repository}

\begin{enumerate}[nosep]
  \item Open a terminal or command prompt.
  \item Navigate to the directory where you want to store the project.
  \item Run:
  \begin{verbatim}
  git clone https://github.com/bgonza214/CS3338-Group9.git
  \end{verbatim}
  \item Open Unity Hub and add the cloned project folder as an existing project.
\end{enumerate}

\subsection{Unity Project Setup}

\begin{enumerate}[nosep]
  \item Open the project in Unity using the correct Unity version (e.g., 2021.3.5f1).
  \item Ensure the following packages are installed or enabled:
  \begin{itemize}[nosep]
    \item XR Plugin Management.
    \item Oculus / OpenXR support for Meta Quest 2.
    \item XR Interaction Toolkit.
  \end{itemize}
  \item Configure build settings:
  \begin{enumerate}[nosep]
    \item Go to \texttt{File $\rightarrow$ Build Settings}.
    \item Select \textbf{Android} as the platform and click \textbf{Switch Platform}.
    \item Add the current scene(s) to the build by clicking \textbf{Add Open Scenes}.
  \end{enumerate}
  \item Configure XR Plugin:
  \begin{enumerate}[nosep]
    \item In \texttt{Project Settings $\rightarrow$ XR Plugin Management}, enable the Oculus or OpenXR plugin for Android.
  \end{enumerate}
  \item Configure graphics and API level:
  \begin{enumerate}[nosep]
    \item In \texttt{Project Settings $\rightarrow$ Player $\rightarrow$ Other Settings}:
    \begin{itemize}[nosep]
      \item Set minimum API level to at least \textbf{Android 6.0 (API level 23)}.
      \item Adjust Graphics APIs order (e.g., Vulkan / OpenGLES3) as required for Quest.
    \end{itemize}
  \end{enumerate}
\end{enumerate}

\subsection{Building the APK for Meta Quest 2}

\begin{enumerate}[nosep]
  \item Connect the Meta Quest 2 to the PC via USB and enable developer mode.
  \item In Unity, go to \texttt{File $\rightarrow$ Build Settings}.
  \item Ensure the correct scenes are included.
  \item Click \textbf{Build} or \textbf{Build and Run} and select an output folder.
  \item Install the generated APK on the Meta Quest 2:
  \begin{itemize}[nosep]
    \item Using \texttt{adb install} from the Android SDK, or
    \item Via SideQuest or other deployment tools.
  \end{itemize}
\end{enumerate}

\section{Running the Training Simulation}

\subsection{Launching the Application}

\begin{enumerate}[nosep]
  \item Put on the Meta Quest 2 headset.
  \item Locate the installed training application in your app library.
  \item Launch the application.
\end{enumerate}

\subsection{Main Menu and Navigation}

On startup, the user is greeted with a simple UI explaining the training simulation.

Typical options include:

\begin{itemize}[nosep]
  \item \textbf{Next}: Proceed to the next area of training.
  \item \textbf{Play Audio}: Play a voice-over for the text instructions currently displayed.
  \item \textbf{Previous}: Go back to the previous training area (available after the first step).
  \item \textbf{Menu / Wrist Menu}: Bring up options such as restarting the simulation,
  jumping to specific scenes, or viewing remaining tasks.
\end{itemize}

\subsection{Controller Basics}

The simulation assumes the user is familiar with basic Meta Quest 2 controller usage:

\begin{itemize}[nosep]
  \item \textbf{Grab}: Hold and release objects.
  \item \textbf{Trigger / Select}: Interact with UI buttons and confirm actions.
  \item \textbf{Primary / Secondary Buttons}: May be mapped to actions such as opening the menu.
  \item \textbf{Joystick}: Used for snap turning or menu navigation if configured.
\end{itemize}

\section{Using the Training Features}

\subsection{Step-by-Step Training UI}

During training:

\begin{itemize}[nosep]
  \item Instructions for the current step are displayed on an in-world UI panel.
  \item The user completes the step following the on-screen description and any audio prompts.
  \item When the system detects that the step is complete, the \textbf{Next} button becomes available.
  \item The \textbf{Previous} button allows revisiting earlier steps when appropriate.
\end{itemize}

\subsection{Inventory System}

The Inventory System Module (ISM) allows the user to carry important tools between scenes:

\begin{itemize}[nosep]
  \item Each inventory slot holds one item and displays a label with the item name.
  \item If an item is dropped outside the allowed boundary, it is returned to a ``dropped item'' slot.
  \item Items can be grabbed from and placed into the inventory using the VR controllers.
\end{itemize}

\subsection{Audio and Feedback}

\begin{itemize}[nosep]
  \item Sound effects and voice-overs provide guidance and immersion.
  \item A dedicated audio toggle button allows turning audio on or off in the UI.
  \item Haptic feedback in the controllers indicates successful interactions (e.g., picking up the correct tool).
\end{itemize}

\subsection{Pole Master LED Sequence Simulation}

A dedicated module simulates the LED states on the Pole Master:

\begin{itemize}[nosep]
  \item Different LEDs (WAN, ERR, I/O, DC) display different colors and blink patterns
        depending on system state (communication mode, error mode, pairing mode,
        connected mode, power modes, etc.).
  \item Certain user actions (e.g., using a magnet tool, completing installation steps)
        trigger changes in the LED sequences.
\end{itemize}

\section{Troubleshooting}

\subsection{Common Issues}

\paragraph{The application does not appear on the headset.}
\begin{itemize}[nosep]
  \item Confirm the APK installed successfully on the Meta Quest 2.
  \item Make sure developer mode is enabled when deploying.
\end{itemize}

\paragraph{The simulation runs slowly or stutters.}
\begin{itemize}[nosep]
  \item Close other intensive applications on the headset.
  \item Ensure you are using the correct build target and graphics settings.
  \item Check that you are using supported API levels and XR plugins.
\end{itemize}

\paragraph{Controllers do not interact with UI or objects.}
\begin{itemize}[nosep]
  \item Verify that XR Interaction Toolkit components are properly attached to controllers.
  \item Ensure that interaction layers and physics colliders are configured correctly.
\end{itemize}

\section{Contributing and Development Workflow}

\begin{enumerate}[nosep]
  \item Create a feature branch from the main branch.
  \item Implement your feature in Unity and commit changes regularly.
  \item Push your branch to the remote repository.
  \item Open a pull request and link it to the relevant Jira ticket.
  \item Request code review from teammates.
  \item Once approved, merge into the main branch.
\end{enumerate}

\section{Credits and Contact}

This project was developed by:

\begin{itemize}[nosep]
  \item Derek Rosales
  \item Brailey Gonzalez-Oxlaj
  \item Kevin Guerrero
\end{itemize}

For questions or support, please contact the project team or the course instructor.

\end{document}