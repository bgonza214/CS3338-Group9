\documentclass[12pt]{article}
\usepackage{geometry}
\usepackage{fancyhdr}
\usepackage{graphicx}
\usepackage{titling}

\geometry{a4paper, margin=1in}

\title{Snapshot Objectives}
\author{
    Brailey Gonzalez, Kevin Guerrero, Derek Rosales
}
\date{December 2025}

\setlength{\droptitle}{6cm}

\pagestyle{fancy}
\fancyhead[L]{Snapshot Objectives}
\fancyhead[R]{Page \thepage}
\fancyfoot[C]{}

\begin{document}

\begin{titlepage}
\maketitle
\thispagestyle{empty}
\end{titlepage}

\section*{Start Objective}
The starting objective of the Southern California Edison VR Training Simulation project is to establish the framework and 
dependencies. During this stage, we will install and configure Unity with Oculus SDK and OpenXR, set up Node.js with Express
to serve as the backend, and initialize a PostgreSQL database to store accounts, sessions, and scores. We will also establish
version control with GitHub to support collaboration. Finally, we will design the basic workflow structure that includes login, 
scenario selection, and score generation, laying the foundation for the rest of the project.

\section*{Checkpoint 1}
The primary objective for Snapshot 2 is to implement the electrical wiring simulation, which represents the core of the program. 
In this stage, we will develop wiring interaction mechanics in Unity so trainees can connect circuits and complete wiring tasks. 
A task display system will be created to present four to five objectives that guide the trainee through the simulation. We will 
also implement login and sign-up functionality to manage user accounts and ensure backend services and the database are fully 
integrated to save trainee progress. Alongside these features, we will add our TestRail and Snapshot Objective document, 
incorporate any new dependencies or libraries required, update the user manual and design specification to reflect the new 
functionality, revise the SDD and SRS with new information, and update the workflow to account for the wiring simulation and login
system.

\section*{Checkpoint 2}
The primary objective for Snapshot 3 is to expand the training scenarios with equipment handling and introduce the scoring system. 
At this stage, trainees will be able to pick up, move, and operate equipment within the VR environment. Backend logging will be 
integrated to record equipment usage events, and the database schema will be extended to store equipment training sessions. A 
scoring system will be implemented to evaluate performance based on accuracy, time taken, and safety compliance, with scores 
displayed at the end of each training session. In addition, we will add our TestRail and Snapshot Objective document, include any 
new dependencies or libraries needed for equipment handling and scoring, update the user manual and design specification, 
revise the SDD and SRS with additional details, and update the workflow to reflect the expanded functionality.

\section*{Final Checkpoint}
The primary objective of Snapshot 4 is to finalize the project and prepare it for submission. This stage will focus on applying 
finishing touches to the VR client and backend, fixing bugs, and optimizing performance. We will complete the TestRail and Snapshot 
Reflection document, add any final dependencies or libraries, update the user manual and design specification for completeness, 
and finalize the SDD and SRS with all project information. The workflow will also be updated one final time to reflect the 
completed system. The last step will be to deliver the final workflow diagrams and documentation package, ensuring that the project 
is polished and ready for evaluation.

\end{document}