\documentclass[11pt]{article}

\usepackage[margin=1in]{geometry}
\usepackage{setspace}
\usepackage{hyperref}
\usepackage{enumitem}
\usepackage{longtable}
\usepackage{graphicx}

% Simple, clean sans-serif font
\usepackage{helvet}
\renewcommand{\familydefault}{\sfdefault}

\setstretch{1.15}

\title{Design Specification\\[0.25em]
SCE VR Training Program}
\author{Brailey Gonzalez-Oxlaj, Derek Rosales ,Kevin Guerrero}
\date{\today}

\begin{document}

\maketitle

\tableofcontents
\newpage

\section{Introduction}

\subsection{Purpose}
This document describes the software design for the SCE VR Training Program.
It explains the overall architecture, key modules, user interface design, and all
dependencies required for the project to run. It also provides a consolidated list
of packages, libraries, and tools that would be needed to containerize the build
and supporting tooling for the project.

\subsection{System Overview}
The system is a virtual reality training application built with the Unity game
engine. It targets the Meta Quest 2 headset and simulates overhead training
scenarios that field workers encounter when working with electric poles and
related equipment. The application provides:

\begin{itemize}[nosep]
  \item A main menu and user interface for navigating training steps.
  \item Interactive VR controls for grabbing, placing, and manipulating tools.
  \item An inventory system for transporting tools between scenes.
  \item Audio management for sound effects and voice-over.
  \item LED sequence simulation for the Pole Master device.
\end{itemize}

Assets such as 3D models are created in Blender and/or Autodesk Maya and then
imported into Unity.

\section{Design Considerations}

\subsection{Assumptions and Dependencies}

The design assumes:

\begin{itemize}[nosep]
  \item Users have a Meta Quest 2 headset with up-to-date firmware.
  \item Users are familiar with basic Meta Quest 2 controls (grab, select, primary/secondary buttons).
  \item Users can read English instructions shown in the UI.
  \item Training will be conducted in a safe indoor environment with sufficient open space.
\end{itemize}

\subsection{User Experience Constraints}

\begin{itemize}[nosep]
  \item The utility truck bucket is static and cannot be moved by the user.
  \item The user teleports between predefined areas instead of using free locomotion.
  \item User interaction with the environment is guided and constrained to relevant objects.
  \item All training guidance is provided via text UI and optional audio.
\end{itemize}

\subsection{System Constraints}

\begin{itemize}[nosep]
  \item The application is designed and tested for Meta Quest 2; other VR headsets are not officially supported.
  \item Performance must be sufficient to maintain a comfortable frame rate for VR (typically 72+ FPS).
\end{itemize}

\subsection{Goals and Guidelines}

\begin{itemize}[nosep]
  \item Provide a user-friendly and straightforward guided training experience.
  \item Base the simulation on real-world procedures and equipment used at SCE.
  \item Increase trainee engagement, retention, and confidence.
  \item Use modular design to facilitate future scenario additions or modifications.
\end{itemize}

\subsection{Development Method}

The team followed an agile development approach with small, iterative tasks,
frequent feedback, and continuous integration of features. Collaboration tools
included Zoom and Discord for meetings, GitHub for source control, and Google
Drive for document sharing.

\section{Architectural Strategies}

\subsection{Tools and Technologies}

\begin{itemize}[nosep]
  \item \textbf{Game Engine:} Unity (2021.3.x LTS).
  \item \textbf{Programming Language:} C\#.
  \item \textbf{IDE:} Microsoft Visual Studio 2019 or later.
  \item \textbf{3D Modeling:} Blender and Autodesk Maya.
  \item \textbf{VR Headset:} Meta Quest 2.
  \item \textbf{Unity Packages:}
  \begin{itemize}[nosep]
    \item XR Plugin Management.
    \item Oculus / OpenXR plugin for Meta Quest 2.
    \item XR Interaction Toolkit.
  \end{itemize}
\end{itemize}

\subsection{Asset Workflow}

\begin{itemize}[nosep]
  \item 3D assets are modeled in Blender or Maya.
  \item Assets are exported and imported into Unity as FBX or similar formats.
  \item Materials, colliders, and interaction components are configured in Unity.
\end{itemize}

\subsection{Coding Guidelines}

\begin{itemize}[nosep]
  \item Scripts include header comments explaining the purpose of the file.
  \item Code sections are commented to clarify logic and interactions.
  \item Functionality is encapsulated into modules with clear responsibilities.
\end{itemize}

\section{System Architecture}

\subsection{High-Level Structure}

At a high level, the system consists of the following major subsystems:

\begin{itemize}[nosep]
  \item Main Menu System.
  \item UI Management.
  \item Inventory System.
  \item Audio Management.
  \item VR Interaction.
  \item Pole Master LED Sequence module.
\end{itemize}

These subsystems interact with each other to form the complete VR training simulation.

\subsection{Level 0 Data Flow Diagram}

Figure~\ref{fig:level0} illustrates the Level 0 data flow, showing the VR Training
Simulation and its main subsystems.

\begin{figure}[h]
  \centering
  % TODO: Replace placeholder with an actual diagram file if desired
  \includegraphics[width=0.7\textwidth]{level0_dfd_placeholder.png}
  \caption{Level 0 Data Flow Diagram}
  \label{fig:level0}
\end{figure}

\subsection{Level 1 Data Flow Diagram}

Figure~\ref{fig:level1} illustrates a more detailed Level 1 data flow, including
scenes, tasks, inventory slots, controls, and user interactions.

\begin{figure}[h]
  \centering
  % TODO: Replace placeholder with an actual diagram file if desired
  \includegraphics[width=0.8\textwidth]{level1_dfd_placeholder.png}
  \caption{Level 1 Data Flow Diagram}
  \label{fig:level1}
\end{figure}

\section{Policies and Tactics}

\subsection{Requirements Traceability}

\begin{itemize}[nosep]
  \item Requirements originate from the SRS, training manuals, and sponsor meetings.
  \item Ambiguous or missing requirements are clarified with SCE stakeholders.
  \item Design decisions and changes are recorded in meetings, stored via shared documents and Jira.
\end{itemize}

\subsection{Testing Strategy}

\begin{itemize}[nosep]
  \item Continuous feature testing during development sessions.
  \item Manual playtesting to validate user experience, interaction correctness, and step progression.
  \item Focus on verifying:
  \begin{itemize}[nosep]
    \item Correct behavior of modules (UI, inventory, VR interaction, audio).
    \item Freedom from critical, simulation-breaking bugs.
    \item Smooth performance on Meta Quest 2 hardware.
  \end{itemize}
\end{itemize}

\section{Detailed System Design}

\subsection{Main Menu Module (MMM)}

\subsubsection{Responsibilities}
The Main Menu Module provides entry points to the training simulation and
allows users to navigate between scenes or restart the training.

\subsubsection{Constraints}
\begin{itemize}[nosep]
  \item Text is currently provided in English.
\end{itemize}

\subsubsection{Composition}
\begin{itemize}[nosep]
  \item Scene selection option to jump to different training steps or scenes.
  \item Tasks option to display a checklist of steps and indicate progress.
  \item Restart option to reset the training simulation from the beginning.
\end{itemize}

\subsubsection{Interactions}
The module works with the VR Interaction module to process button presses and
updates the UI Management and Inventory modules based on user choices.

\subsection{UI Management Module (UIMM)}

\subsubsection{Responsibilities}
The UI Management Module provides step-by-step text instructions and controls
to guide the user through the training flow.

\subsubsection{Constraints}
\begin{itemize}[nosep]
  \item Users must read English instructions.
  \item Users must understand basic VR controller operations.
\end{itemize}

\subsubsection{Composition}
\begin{itemize}[nosep]
  \item Text area describing the current step's goal.
  \item \textbf{Next} button (enabled only when the current step is completed).
  \item \textbf{Previous} button (where applicable).
  \item Audio toggle/play button for voice-over or sound effects.
\end{itemize}

\subsubsection{Interactions}
The module monitors completion conditions reported by VR Interaction and other
modules to determine when the user may proceed. It also triggers audio via the
Audio Management Module and receives input from the controllers.

\subsection{Inventory System Module (ISM)}

\subsubsection{Responsibilities}
The Inventory System Module handles the logic for storing, retrieving, and
managing tools and objects used across the training scenes.

\subsubsection{Constraints}
\begin{itemize}[nosep]
  \item Inventory size is limited to a small number of relevant items.
\end{itemize}

\subsubsection{Composition}
\begin{itemize}[nosep]
  \item Inventory slot UI elements.
  \item Text labels showing the name of the object in each slot.
  \item Dropped item slot that receives items dropped outside the valid area.
\end{itemize}

\subsubsection{Interactions}
Users use VR controllers (via the VR Interaction Module) to transfer items between
the environment and inventory slots. Items dropped outside bounds are teleported
back to the inventory.

\subsection{Audio Management Module (AMM)}

\subsubsection{Responsibilities}
The Audio Management Module organizes and plays sound effects and voice-over
clips. It ensures consistent volume levels and 3D audio positioning.

\subsubsection{Constraints}
\begin{itemize}[nosep]
  \item Supported formats include AIFF, WAV, MP3, and Ogg.
  \item Audio data may be preloaded or streamed as required.
\end{itemize}

\subsubsection{Composition}
\begin{itemize}[nosep]
  \item Global audio manager object controlling volume and mixer settings.
  \item Audio profiler tools for verifying audio memory and performance.
\end{itemize}

\subsubsection{Interactions}
The AMM is triggered by UI events (e.g., audio buttons), VR interaction events
(e.g., correct item usage), and scene transitions to play appropriate sounds.

\subsection{VR Interaction Module (VRIM)}

\subsubsection{Responsibilities}
The VR Interaction Module handles the mapping between user inputs (controllers)
and actions in the virtual environment. It is responsible for:

\begin{itemize}[nosep]
  \item Grabbing and releasing objects.
  \item Teleportation between valid locations.
  \item Interacting with UI elements via ray-based pointers.
  \item Snap turning and other movement/rotation options.
\end{itemize}

\subsubsection{Constraints}
\begin{itemize}[nosep]
  \item Performance must remain high to avoid VR sickness.
  \item The user generally interacts with one object at a time per hand.
\end{itemize}

\subsubsection{Composition}
\begin{itemize}[nosep]
  \item Interactable components (objects that respond to grab/use).
  \item Grabbable components for movable items.
  \item Teleportation component for location changes.
  \item UI Interactor for VR-based UI clicking.
  \item Snap Turn component for discrete rotation.
\end{itemize}

\subsubsection{Interactions}
It works closely with UI Management, Inventory, Audio, and the LED Sequence
module. Controller rays or colliders trigger events such as button presses, object
pickup, and step completion detection.

\subsection{Pole Master LED Sequence Module (PMLSM)}

\subsubsection{Responsibilities}
This module simulates Pole Master LED behavior, including WAN, ERR, I/O, and
DC LEDs, according to the state of the training scenario.

\subsubsection{Constraints}
Each LED must display the correct color and blinking pattern for different modes
(e.g., communication, error, pairing, connected, stable power, backup battery).

\subsubsection{Composition}
\begin{itemize}[nosep]
  \item WAN LED:
  \begin{itemize}[nosep]
    \item Communication Mode: constant blue.
  \end{itemize}
  \item ERR LED:
  \begin{itemize}[nosep]
    \item Error Mode: red flash once per second.
  \end{itemize}
  \item I/O LED:
  \begin{itemize}[nosep]
    \item Pairing Mode: yellow flash twice per second.
    \item Connected Mode: yellow flash once per second.
  \end{itemize}
  \item DC LED:
  \begin{itemize}[nosep]
    \item Stable Power Mode: constant green.
    \item Backup Battery Mode: green flash once per second.
  \end{itemize}
\end{itemize}

\subsubsection{Interactions}
Scripts attached to the Pole Master and related tools (such as the magnet tool)
trigger LED state changes. Completion of smart navigator installation, or specific
user actions, causes the LEDs to update.

\section{User Interface Design}

\subsection{Overview}

The UI is designed to be as simple and readable as possible. Upon launching the
application, users see a panel that:

\begin{itemize}[nosep]
  \item Introduces the training scenario.
  \item Provides ``Next'' and ``Play Audio'' buttons.
\end{itemize}

Subsequent UI panels include:

\begin{itemize}[nosep]
  \item ``Previous'' and ``Next'' buttons for step navigation.
  \item ``Play Audio'' button for voice-over.
  \item Optional access to a menu with options like restart, scene selection, and task list.
\end{itemize}

\subsection{Screen Frameworks}

% TODO: Replace placeholders with actual UI screenshots
Figure~\ref{fig:ui_screens} shows example UI screens from the training application.

\begin{figure}[h]
  \centering
  \includegraphics[width=0.8\textwidth]{ui_screens_placeholder.png}
  \caption{Example User Interface Screens}
  \label{fig:ui_screens}
\end{figure}

\subsection{UI Flow Model}

The UI flow consists of a linear progression of training steps, each with navigation
buttons and optional menu access. A simple flow model can be represented as:

\begin{itemize}[nosep]
  \item User Interface
  \begin{itemize}[nosep]
    \item Next Button
    \item Previous Button
    \item Audio Button (connected to Audio Management)
    \item Menu Button (connected to Main Menu Module)
  \end{itemize}
\end{itemize}

\section{Requirements Validation and Verification (Mapping)}

Table summarizes how key functional requirements are
satisfied by modules and UI elements.

\begin{longtable}{p{0.35\textwidth} p{0.55\textwidth}}
\hline
\textbf{Functional Requirement} & \textbf{Modules / UI Elements} \\
\hline
System shall run smoothly on target hardware &
Optimized component modules and lightweight UI; performance testing on Meta Quest 2. \\
System should be free of system-breaking errors/bugs &
Robust module interactions; continuous testing and play sessions. \\
System shall have sound effects and voice-over &
Audio Management Module, audio UI buttons. \\
System shall have options menu &
Options menu within Main Menu Module and UI Management. \\
System shall allow player to interact with objects in environment &
VR Interaction Module, interactable/grabbable components, inventory system. \\
System shall allow user to adjust volume &
Volume controls exposed via audio UI and Audio Management Module. \\
\hline
\end{longtable}

\section{Dependencies and Environment}

\subsection{Runtime Dependencies}

At runtime on the Meta Quest 2 device, the application depends on:

\begin{itemize}[nosep]
  \item Meta Quest 2 OS (version 46.0 or higher).
  \item Android 6.0 (API level 23) or higher.
  \item Oculus / OpenXR runtime for VR.
  \item Audio codecs for AIFF, WAV, MP3, and Ogg playback.
\end{itemize}

\subsection{Development Environment}

On the development machine, the following are required:

\begin{itemize}[nosep]
  \item Windows 10 or later (22H2 recommended).
  \item Unity Editor 2021.3.x LTS with:
  \begin{itemize}[nosep]
    \item Android Build Support (SDK, NDK, OpenJDK).
    \item XR Plugin Management.
    \item Oculus / OpenXR support.
    \item XR Interaction Toolkit.
  \end{itemize}
  \item Unity Hub.
  \item Visual Studio 2019 or later with C\# support.
  \item Git (for version control).
  \item Optional: Blender and Autodesk Maya for asset creation.
\end{itemize}

\subsection{Unity Packages and Libraries}

Key Unity packages and libraries include:

\begin{itemize}[nosep]
  \item \textbf{XR Interaction Toolkit}: Provides core VR interaction components.
  \item \textbf{XR Plugin Management}: Manages underlying VR runtimes.
  \item \textbf{Oculus / OpenXR plugin}: Handles Meta Quest 2 integration.
  \item \textbf{TextMeshPro} (if used): For higher-quality text rendering.
  \item Custom C\# scripts for:
  \begin{itemize}[nosep]
    \item Step management and progression.
    \item Inventory logic.
    \item Audio triggering.
    \item LED state control.
  \end{itemize}
\end{itemize}

\section{Docker and Containerization Considerations}

\subsection{Purpose of Docker in This Project}

Because VR applications must run on hardware with direct access to the headset,
Docker is primarily useful for:

\begin{itemize}[nosep]
  \item Automated build pipelines (CI/CD) for the Unity project.
  \item Running unit or integration tests that do not require an actual headset.
  \item Hosting project assets, logs, or documentation.
\end{itemize}

This section lists the packages and tools that would be needed if a Docker
container is used to build or process the project.

\subsection{Base Image and System Packages}

A typical container for Unity build automation might include:

\begin{itemize}[nosep]
  \item Base image: \texttt{ubuntu:20.04} or similar.
  \item System packages (installed via \texttt{apt}):
  \begin{itemize}[nosep]
    \item \texttt{curl}, \texttt{wget}, \texttt{git}, \texttt{unzip}.
    \item \texttt{ca-certificates}, \texttt{tzdata}.
    \item \texttt{libgtk-3-0}, \texttt{libnss3}, \texttt{libx11-6}, \texttt{libasound2}
          (commonly required for Unity CLI).
    \item \texttt{python3} (if helper scripts are used).
  \end{itemize}
\end{itemize}

\subsection{Unity Editor and Build Tools}

\begin{itemize}[nosep]
  \item Unity Editor (Linux headless / CLI version matching 2021.3.x).
  \item Android SDK, NDK, and OpenJDK (for Android builds).
  \item Environment variables configured for Unity license activation (in CI) and
        Android build tools.
\end{itemize}

\subsection{Project-Specific Dependencies in the Container}

Within the container, you would copy:

\begin{itemize}[nosep]
  \item Full Unity project directory (Assets, ProjectSettings, Packages, etc.).
  \item Custom C\# scripts for all modules (MMM, UIMM, ISM, AMM, VRIM, PMLSM).
  \item Any required asset packages that are not fetched automatically (or configure Unity
        to download them from the Unity Asset Store).
\end{itemize}

If the container is strictly used for building the APK, VR headset access is not
required; only the build tools and project content are needed.

\section{Glossary}

\begin{description}[style=nextline]
  \item[Blender] 3D modeling tool used to create models, simulations, and animations.
  \item[C\#] High-level programming language used for scripting in Unity.
  \item[Discord] Communication platform used for text, voice, and screen sharing.
  \item[GitHub] Web-based platform for hosting Git repositories and collaborating on code.
  \item[Google Drive] Cloud storage service used for storing and sharing documents.
  \item[Meta Quest 2] Standalone VR headset used to run the training application.
  \item[Unity] Game engine used to create 2D/3D applications, including this VR training app.
  \item[Virtual Reality (VR)] Technology that simulates 3D environments that users can interact with.
  \item[Zoom] Video conferencing tool used for meetings and remote collaboration.
\end{description}

\section{References}

\begin{itemize}[nosep]
  \item Unity and VR tutorials, XR Interaction Toolkit documentation.
  \item YouTube resources on VR interaction, inventory systems, sockets, triggers, and UI in Unity.
  \item SCE training manuals and in-person training observations.
\end{itemize}

\end{document}