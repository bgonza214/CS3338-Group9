\documentclass[12pt]{article}
\usepackage{geometry}
\usepackage{fancyhdr}
\usepackage{graphicx}
\usepackage{titling}
\usepackage{hyperref}

\geometry{a4paper, margin=1in}

\title{Southern California Edison VR Training Simulation\\
Software Design Document}
\author{Brailey Gonzalez, Kevin Guerrero, Derek Rosales}
\date{December 2025}

\setlength{\droptitle}{6cm}

\pagestyle{fancy}
\fancyhead[L]{SCE VR Training Simulation SDD}
\fancyhead[R]{Page \thepage}
\fancyfoot[C]{}

\begin{document}

\begin{titlepage}
\maketitle
\thispagestyle{empty}
\end{titlepage}

\thispagestyle{empty}
\tableofcontents
\newpage

\section*{Version History}
\addcontentsline{toc}{section}{Version History}
\begin{table}[ht]
    \centering
    \begin{tabular}{|c|c|c|c|}
    \hline
    \textbf{User} & \textbf{Date} & \textbf{Reason for Changes} & \textbf{Version}\\
    \hline
    Brailey Gonzalez & 12/5/25 & Snapshot 1 Update & 1.0 \\
    \hline
    Kevin Guerrero & 12/6/25 & Snapshot 2 Update & 2.0 \\
    \hline
    Derek Rosales & 12/7/25 & Snapshot 3 Update & 3.0 \\
    \hline
    Team & 12/9/25 & Snapshot 4 Update & 4.0 \\
    \hline
    \end{tabular}
\end{table}
\newpage

\section{Introduction}
\subsection{Purpose}
This Software Design Document (SDD) serves as a blueprint for the creation of the Worker Training Simulation (WTS) program.  
It provides a structured plan that outlines the system’s architecture, core functionality, and design decisions.  
The document is intended to guide the development team by clearly defining how the VR training environment will be built,  
how users will interact with the system, and how data will be managed.

\subsection{Scope}
The scope of this document includes the design and implementation of the WTS.  
It covers the VR framework setup, database integration, and task delegation for the development team.

\subsection{Intended Audience}
The intended audience for this document includes the development team, and users who will specifically be trainees who perform field work such as electrical wiring, equipment handling, and safety tasks.

\subsection{Overview}
The WTS application will be a VR-based simulation that allows workers to practice tasks safely in a virtual environment.  
The application will provide a user-friendly interface and will be designed to improve safety awareness, task proficiency, and experience.

\subsection{References}
See the references section for a list of documents and resources referenced in this document.

\subsection{Definitions, Acronyms, and Abbreviations}
See the glossary section for definitions of terms, acronyms, and abbreviations used in this document.
\newpage

\section{System Architecture}
\subsection{Workflow}
\begin{itemize}
    \item Trainee logs in through the VR client using their credentials.
    \item Trainee selects a training scenario.
    \item Trainee performs tasks such as wiring tasks and equipment management.
    \item The VR client logs trainee actions and sends them to the backend.
    \item The backend stores data in the PostgreSQL database.
    \item Scores are generated and stored for later review.
\end{itemize}

\subsection{Site Breakdown}
\begin{itemize}
    \item \textbf{VR Client:} Developed in Unity with Oculus SDK and OpenXR support.
    \item \textbf{Backend Services:} Implemented in Node.js with Express for REST APIs.
    \item \textbf{Database:} PostgreSQL database for storing trainee accounts, scenarios, and training scores.
    \item \textbf{Version Control:} GitHub repository for collaboration and code management.
\end{itemize}

\subsection{Architecture Overview}
The VR client follows a Model-View-Controller (MVC) architectural pattern:
\begin{itemize}
    \item \textbf{Model:} Represents training data and user progress.
    \item \textbf{View:} VR environment and user interface.
    \item \textbf{Controller:} Handles interactions and communication with backend APIs.
\end{itemize}

\subsection{Data Flow}
\begin{enumerate}
    \item The trainee interacts with objects in the VR training room.
    \item The VR client logs actions and sends them to the backend via REST API.
    \item The backend stores data in PostgreSQL.
    \item Scores are calculated and stored for each trainee session.
\end{enumerate}
\newpage

\section{User Interface}
\subsection{How to Use}
\subsubsection{VR Client}
Launch the VR application on a supported headset. The main menu contains options for:
\begin{itemize}
    \item \textbf{Begin Training:} Enter the simulation.
    \item \textbf{Select Training}
    \item \textbf{Perform Tasks}
    \item \textbf{Receive Score:} Score is displayed when training is complete.
    \item \textbf{Exit:} Close the application.
\end{itemize}

Inside the simulation:
\begin{itemize}
    \item Trainees can pick up, move, and inspect objects.
    \item Feedback is provided through object highlights and end-of-session scores.
\end{itemize}

\subsection{Database Explanation}
The database for the WTS is implemented using PostgreSQL.  
It stores all data related to trainees and their training sessions, including:
\begin{itemize}
    \item \textbf{Trainee Accounts:} Login credentials (username, password).
    \item \textbf{Profiles:} Basic trainee details such as name and department.
    \item \textbf{Training Sessions:} Records of each session including scenario ID, start/end time, and completion status.
    \item \textbf{Scores:} Performance scores generated from each training session.
    \item \textbf{Scenario Definitions:} Stored procedures, hazards, and scoring rules for training modules.
\end{itemize}

Database management tools such as pgAdmin and DBeaver will be used to administer PostgreSQL.
\newpage

\section*{Glossary}
\addcontentsline{toc}{section}{Glossary}
\begin{description}
    \item[\textbf{WTS}] - Worker Training Simulation
    \item[\textbf{VR}] - Virtual Reality
    \item[\textbf{SDK}] - Software Development Kit
    \item[\textbf{UI}] - User Interface
    \item[\textbf{MVC}] - Model-View-Controller
    \item[\textbf{PostgreSQL}] - Relational database management system used for WTS
\end{description}
\newpage

\addcontentsline{toc}{section}{References}
\begin{thebibliography}{9}
    Ascent - Project (2022). \url{https://ascent.cysun.org/project/project/view/187}
\end{thebibliography}

\end{document}